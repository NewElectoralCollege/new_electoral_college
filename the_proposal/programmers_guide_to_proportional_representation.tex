\documentclass{article}
\usepackage[utf8]{inputenc}
\usepackage{listings}
\usepackage{xcolor}
\usepackage{geometry}

\definecolor{codegreen}{rgb}{0,0.6,0}
\definecolor{codepurple}{rgb}{0.58,0,0.82}

\geometry{legalpaper, portrait, margin=1in}

\lstdefinestyle{coding}{  
    commentstyle=\color{codegreen},
    keywordstyle=\color{blue},
    numberstyle=\tiny\color{gray},
    stringstyle=\color{codepurple},
    basicstyle=\ttfamily\footnotesize,
    breakatwhitespace=false,
    breaklines=false,
    captionpos=b,   
    keepspaces=true,
    numbers=left,   
    numbersep=5pt, 
    showspaces=false,       
    showstringspaces=false,
    showtabs=false, 
    tabsize=1
}

% Coding examples can be found here: https://github.com/KingWither/new_electoral_college/tree/main/the_proposal/programming_examples

\lstset{style=coding}

\title{A Programmer's Guide to Proportional Representation}
\author{David G. Boers}

\begin{document}
    \maketitle
    
    \section{Introduction}

    Proportional Representation is best calculated on the computer. It is more efficient and more accurate than doing it by hand. This will contain three implementations of Proportional Representation.

    \section{Python}

    The results displayed on newelectoralcollege.com were computed using Python. Below are the basics of the implementation.

    \lstinputlisting[language=Python, firstline=160, lastline=181]{../makeResults.py}

    Python is very easy to read, meaning even people who aren't intimatly familiar with the language will be able to read it as Pseudocode. This makes Python the ideal langugage to write a calculator in, especialy if it needs to be released to the public.\\

    That implementation is designed for the specific purpose of the New Electoral College's website. Below is another version created just for example.

    \lstinputlisting[language=Python]{programming_examples/python.txt}

    \section{C++}

    \lstinputlisting[language=C++]{programming_examples/c++.txt}

    \section{Purely Functional}

    Haskell is amoung the most widley used Purely Functional languages.

    \lstinputlisting[language=Haskell]{programming_examples/haskell.txt}

    Here is a slight modification of the Haskell implementation with other quotas.

    \lstinputlisting[language=Haskell]{programming_examples/haskell_quotas.txt}

    The implementation used for the Result Calculator on newelectoralcollege.com is written in Elm. This is not the same as the Python implementation above. That is used to calculate results in specific Presidential elections under The New Electoral College. This is used for the example calculator.

    \lstinputlisting[language=Haskell]{../src/elm/Calculator/Hare.elm}

    The lambdaCompare function is defined as this:

    \begin{lstlisting}
    lambdaCompare : (a -> a -> Bool) -> a -> (b -> a) -> b -> Bool
    lambdaCompare comp value function record =
        comp (function record) value
    \end{lstlisting}

\end{document}