\documentclass{article}
\usepackage{amsmath}
\usepackage[utf8]{inputenc}
\usepackage{dirtytalk}
\title{The New Electoral College}
\author{David Boers}

\begin{document}
    \maketitle
    A quick search on the internet will bring up hundreds of hits on the Electoral College. How it works, why the Founding Fathers chose it, and, 
    more recently, why it should be abolished. It should be no surprise that interest in the Electoral College is so high; it chooses our President. 
    But like all of the institutions that were born when the Constitution was ratified in 1788, it needs updating. The update that we have proposed is 
    one that we believe addresses many of the concerns that people have with the way the Electoral College works, but our proposal mostly works towards 
    making every vote count.

    Firstly, here is a fact that you may not have known: There really is no such thing as a United States Presidential Election. The "Election", which 
    really just consists of 51 individual contests, was not intended to be the Presidential selection process. The Founders didn’t really envision a 
    Popular Election for President in any way. Instead, the Constitution says:

    \say{Each State shall appoint, in such Manner as the Legislature thereof may direct, a Number of Electors, equal to the whole Number of Senators and Representatives to which the State may be entitled in the Congress: but no Senator or Representative, or Person holding an Office of Trust or Profit under the United States, shall be appointed an Elector.}

    This suggests that each state determines how to pick Electors. In the first Presidential Election, in 1788, only 6 of the 13 states had a Popular Vote 
    for President. It was not until 1836 that all Electors were chosen by Election. Today, 48 of the 50 states use the "General Ticket" method, also known 
    as "Winner-take-All." The two that don't, Maine and Nebraska, use a very similar method, involving the states' Congressional Districts.

    This means that if you live in a very blue state, but are a Republican, your vote doesn't really count. The Republican candidate might win the 
    election, but your state chose different Electors. This works conversely with Democrats. It also means that a candidate who narrowly won some small 
    states might end up with more electoral votes than a candidate who won every vote in another state.

    Our proposal is to get rid of the General Ticket method for choosing Presidential Electors. Instead, Electors would be chosen using a method of 
    Proportional Representation.

    \textbf{What is Proportional Representation?}
    Many countries, including the United States, use the First-Past-the-Post election system for their legislators. This involves the person with the 
    most votes winning the election. Other countries use a Proportional Representation system, where seats are awarded with the same proportion of seats 
    as votes. For instance, if you get 50\% of the votes in a constituency with 10 seats, you get 5 seats.

    There are many ways of achieving Proportional Representation. This will explain the method that we have chosen for the Electoral College. First, you 
    determine a quota that represents 1 seat. There are different quotas, each producing different results in different circumstances, but I have chosen 
    the simplest: the Hare Quota. We determine the Hare Quota by dividing the total number of votes cast in a state by the number of Electoral Votes the
    state gets.

    \begin{align}
        h &= \frac{votes}{electors}
    \end{align}

    If 1,000,000 votes are cast in a 10 Elector-state, the quota is 100,000. If you get 275,000 votes, you initially get 2 Electors. This means that 
    75,000 votes are left. Any Electors that can not be awarded by full quotas get allocated based off this remainder. If there are 2 extra Electors to 
    be awarded, and your surplus of 75,000 is one of the top 2 surpluses, you get an extra Elector.

    \textbf{It's Fair}

    The New Electoral College will bring fairness and ballance to our Presidential Election. In the 2020 Presidential Election, 91,553,753 people voted 
    for a candidate that either lost their state, or their candidate had already won their state, so their ballot didn't count. That is 58.8\% of the 
    total votes. Under our proposed system, only 12,414,959 (7.8\%) of the votes would be wasted.

    \textbf{It's Ballanced}

    The effect of the Electoral College is not diminished by this proposal. Every state will still have their say in the Presidency, and won't be 
    silenced by larger states, while ensuring that each individual ballot counts, even in very lopsided states.

    \textbf{Why Not Abolish?}

    The Electoral College is important. It requires a consensus to be gained that a candidate would make a good President. You need people of all kinds 
    of backgrounds to agree to elect you.

    The Economist Intelligence Unit (EIU) publishes a "Democracy Index," ranking countries based on factors like Electoral process, Political 
    participation, and Civil liberties. Of the countries in the top catagory of this index, the so-called Full Democracies, none have a directly elected 
    National Executive. Three, Iceland, Ireland, and Finland, have a directly elected Presidency, but in those cases the Presidents don't execute 
    authority. On the other side, 80\% of the countries ranked as Authoritarian have directly elected leaders with power.
\end{document}