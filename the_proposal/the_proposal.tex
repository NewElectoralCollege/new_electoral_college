\documentclass{article}
\usepackage{amsmath}
\usepackage[utf8]{inputenc}
\usepackage{dirtytalk}
\usepackage{adjustbox}
\title{The New Electoral College}
\author{David G. Boers}

\begin{document}
    \maketitle
    The Electoral College of the United States is the body that chooses the President. It has performed this function every 4 years since 1799. Because of the power which the process holds, it has generated intense intrigue. Proposals related to the Electoral College have ranged from reform to abolition, and there has been a wide variety of arguments in favor of both. The proposal which I find most advisable, is the one I call "The New Electoral College." It is of the former type mentioned above.
    The Constitution contains no instructions for how the electors are to be appointed, instead directing the states to set regulations on this matter.\\

    \say{Each State shall appoint, in such Manner as the Legislature thereof may direct, a Number of Electors, equal to the whole Number of Senators and Representatives to which the State may be entitled in the Congress: but no Senator or Representative, or Person holding an Office of Trust or Profit under the United States, shall be appointed an Elector.}\\

    In the first Presidential Election, held during both 1788 and 1789, only 5 of the 13 states held an election for Presidential electors. In the rest, the state legislature themselves appointed the electors. Given that George Washington was unnaposed, and that only 39,624 votes voted at all, representing 11.6\% of the eligible voters, the election was not truley democratic. It was not until 1872 that all electors were chosen by a popular vote. The following table shows the movement by states over time of how they chose electors.\\
    \begin{adjustbox}{center}
    \begin{tabular}{ |l|l|l|l|l| }
        \multicolumn{5}{|l|}{Number of electors chosen by each method over time} \\
        \hline
        Election & At-Large & District\footnote{Included congressional districts, which nessecitated two of the electors being chosen state-wide. In other cases, special districts were drawn.} & Hybrid & Legislative Selection \\
        \hline
        1788-1789 & 18 & 15 & 15 & 33 \\
        1792 & 25 & 25 & 22 & 63
    \end{tabular}
    \end{adjustbox} \\

    Over time, states began to drift away from legislative selection as a method of appointing electors to popular election. Hence, the "United States Presidential Election" was created by de facto, and the misconception of the President being a directly elected position began. Perhaps just as importantly, the laws regarding popular votes for President changed. In the first several elections, just as many or more states used a district-based method to allocate electors as at-large. Over time, states switched to an at-large method. By 1836, no states used the district method. In 1892, the State of Michigan used the method. The reasons for this were entierly partisan. The Democratic controlled legislature wanted to boost former President Grover Cleveland's chances in the state, but felt unsure that he could win it. Cleveland recieved 5 votes, while Republican Benjamin Harrison recieved 9. In 1972, Maine changed to the district based system, and in 1992, Nebraska joined it. \\

    Apart from partisan politics, what are the benefits to a district system? A state that allocates it's electors by district has a higher change of appointing electors that accuratly represent it's electorate. In the 1892 election in Michigan, had the electors been appointed to the winner, Harrison would have won all 14 electors, despite only getting 47.79\% of the vote. Had he won 99.99\%, or by only a single vote, this wouldn't have changed. This renders a significant chunk of the votes irrelivant. \\

    While the district system is a massive improvement over the at-large one, it also doesn't neccisarily represent the voters. Not only is the system vulnerable to Gerrymandering, two electors must still be awarded state-wide. This contributes to an imballace just the same as in at-large. \\

    A clearer solution would be the system contained in this proposal. Under the New Electoral College, electors would be appointed state-wide, but the ratio between electors pledged to a candidate would be approxamily the same as the ratio of votes cast in an election. This is achived using a family of electoral systems, known as Proportional Representation. \\

    The first proposals for an assembly elected through proportionality was made by John Adams. Others began to speak of the idea around the turn of the eighteenth century. The first concrete proposals were introduced by Thomas Hare, an English lawyer. His original ideas included combining the entire Unted Kingdom into one huge electorate. He also devised the original Proportional Representation calculations, which will now be explained. Because Proportional Representation was originaly designed for legislatures, that is how the explination will be written. \\

    In Proportional Representation, true and exact proportionality is not possible. If all votes directly contriuted towards a seat, in other words, an exact ratio, all participants would win at least one seat. Additionaly, there is a finite amount of seats to be awarded, so there is a limited number of combinations of seat allocations. \\

    Hare's original method has several names, including the Hare-Niemeyer method, Largest remainder method, Hamilton's method, and Vinton's method. The Largest remainder method (LRM), is in fact, a family of methods, and the latter two are named after American Founding Father Alexander Hamilton and Representative Samuel Finley Vinton respectivly, both of whom supported the method. \\

    When performing an allocation under Proportional Representation, a quota is calculated to determine how many votes will entitle a party to a seat. Hare's quota is represented as:

    \begin{align}
        \frac{\text{total valid poll}}{\text{seats}}
    \end{align}
    but more presicly:
    \begin{align}
        \left \lfloor \frac{\text{total valid poll}}{\text{seats}} \right \rfloor
    \end{align}

    The \textit{total valid poll} is the number of votes cast in an election that count. This excludes ballots where more than one vote was cast, or where no option was marked. The $\lfloor \rfloor$ markers mean the number is rounded down to the nearest integer.\\

    A party will win a seat for every time their vote count can be divided evenly into the quota. Any extra seats will be awarded to each party in descending order of the party's remainder. The following example election displays the calculation process. In this election, there are 11 seats to be filled, and a total of 1,650,799 votes are cast. Therefore, the quota is 150,072.\\

    \begin{adjustbox}{center}
    \begin{tabular}{ |l|l|l|l|l|l|l| }
        \hline
        Party & Votes & Divisor & Initial Seats & Remainder (Rank) & Extra Seats & Total Seats \\
        \hline
        A & 871,700 & 5.81 & 5 & 0.81 (2) & 1 & 6 \\
        \hline
        B & 739,894 & 4.93 & 4 & 0.93 (1) & 1 & 5 \\
        \hline
        C & 39,205 & 0.26 & 0 & 0.26 (3) & 0 & 0 \\
        \hline
    \end{tabular}
    \end{adjustbox}\\
    
    9 seats may be awarded based on the quota, leaving 2 seats remaining. Party B has the highest remainder, and gets the first extra seat. Party A has the next highest, and so also gets an extra seat. \\

    When the vote percentages are compared with the seat percentages, the results approxamitly align.\\
    
    \begin{adjustbox}{center}
    \begin{tabular}{ |l|l|l| }
        \hline
        Party & Vote Percentage & Seat Percentage \\
        \hline
        A & 52.8\% & 54.55\% \\
        \hline
        B & 44.82\% & 45.45\% \\
        \hline
        C & 2.37\% & 0\% \\
        \hline
    \end{tabular}
    \end{adjustbox}\\

    \section{An important caveat}

    In most countries, the votes are counted by party. That is the way the votes are cast, so it is most accurate to award them that way. However, this fails in the Electoral College. The following example is from the 2016 election in New York.\\

    \begin{adjustbox}{center}
    \begin{tabular}{ |l|l|l|l|l|l|l|l| }
        \hline
        Party & Nominee & Votes & Divisor & Initial Seats & Remainder (Rank) & Extra Seats & Total Seats \\
        \hline
        Democratic & Hillary Clinton & 4,379,804 & 16.45 & 16 & 0.45 & 1 & 17 \\
        \hline
        Republican & Donald Trump & 2,527,164 & 9.49 & 9 & 0.49 & 1 & 10 \\
        \hline
        Conservative & Donald Trump & 292,393 & 1.10 & 1 & 0.10 & 0 & 1 \\
        \hline
        Working Families & Hillary Clinton & 140,044 & 0.53 & 0 & 0.53 & 1 & 1 \\
        \hline
        Independence & Gary Johnson & 119,162 & 0.45 & 0 & 0.45 & 0 & 0 \\
        \hline
        Green & Jill Stein & 107,937 & 0.41 & 0 & 0.41 & 0 & 0 \\
        \hline
        Libertarian & Gary Johnson & 57,438 & 0.22 & 0 & 0.22 & 0 & 0 \\
        \hline
        Women's Equality & Hillary Clinton & 36,292 & 0.14 & 0 & 0.14 & 0 & 0 \\
        \hline
    \end{tabular}
    \end{adjustbox} \\

    New York has rather unique laws that allow a candidate to be nominated by more than one party. This is known as fusion voting. Allocating seats by party under this method is impractical, because it doesn't reflect the vote for President. The following figure shows the results calculated for the same election by candidate. In the example, the votes for each candidate is the sum of votes for all parties that nominated that candidate.\\

    \begin{adjustbox}{center}
    \begin{tabular}{ |l|l|l|l|l|l|l| }
        \hline
        Party & Votes & Divisor & Initial Seats & Remainder (Rank) & Extra Seats & Total Seats \\
        \hline
        Hillary Clinton & 4,556,124 & 17.11 & 17 & 0.11 (4) & 0 & 17 \\
        \hline
        Donald Trump & 2,819,534 & 10.59 & 10 & 0.59 (2) & 1 & 11 \\
        \hline
        Gary Johnson & 176,598 & 0.66 & 0 & 0.66 (1) & 1 & 1 \\
        \hline
        Jill Stein & 107,934 & 0.41 & 0 & 0.41 (3) & 0 & 0 \\
        \hline
    \end{tabular}
    \end{adjustbox} \\

    Johnson is allocated a seat that was assigned to Clinton in the party-wise election. This should be the prefered method of allocating electors, because it relfects the vote for President; The Electoral College is not a legislature, where parties are elected.\\
    
    After the allocation of seats to candidates, it is still neccisary to allocate seats to parties. The Largest-Remainder calculation is performed again, between the parties that nominated each candidate.

\end{document}