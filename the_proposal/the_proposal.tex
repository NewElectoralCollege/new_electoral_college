\documentclass{article}
\usepackage{amsmath}
\usepackage[utf8]{inputenc}
\usepackage{dirtytalk}
\usepackage{adjustbox}
\title{The New Electoral College}
\author{David G. Boers}

\begin{document}
    \maketitle
    The Electoral College of the United States is the body that chooses the President. It has performed this function every 4 years since 1799. Because of the power which the process holds, it has generated intense intrigue. Proposals related to the Electoral College have ranged from reform to abolition, and there has been a wide variety of arguments in favor of both. The proposal which I find most advisable, is the one I call "The New Electoral College." It is of the former type mentioned above.\\
    
    The Constitution contains no instructions for how the electors are to be appointed, instead directing the states to set regulations on this matter.\\

    \say{Each State shall appoint, in such Manner as the Legislature thereof may direct, a Number of Electors, equal to the whole Number of Senators and Representatives to which the State may be entitled in the Congress: but no Senator or Representative, or Person holding an Office of Trust or Profit under the United States, shall be appointed an Elector.}
    \begin{flushright}\textit{--- Constitution of the United States, Article 2, Section 1}\end{flushright}%%\\

    In the first Presidential Election, held during both 1788 and 1789, only 5 of the 13 states held an election for Presidential electors. In the rest, the state legislature themselves appointed the electors. Given that George Washington was unopposed, and that only 39,624 votes voted at all, representing 11.6\% of the eligible voters, the election was not truly democratic. It was not until 1872 that all electors were chosen by a popular vote. The following table shows the movement by states over time of how they chose electors.\\

    \begin{adjustbox}{center}
    \begin{tabular}{ |l|l|l|l|l| }
        \multicolumn{5}{|l|}{Number of electors chosen by each method over time} \\
        \hline
        Election & At-Large & District\footnote{Included congressional districts, which nessecitated two of the electors being chosen state-wide. In other cases, special districts were drawn.} & Hybrid & Legislative Selection \\
        \hline
        1788-1789 & 18 & 15 & 15 & 33 \\
        1792 & 25 & 25 & 22 & 63
    \end{tabular}
    \end{adjustbox} \\

    Over time, states began to drift away from legislative selection as a method of appointing electors to popular election. Hence, the "United States Presidential Election" was created by de facto, and the misconception of the President being a directly elected position began. Perhaps just as importantly, the laws regarding popular votes for President changed. In the first several elections, just as many or more states used a district-based method to allocate electors as at-large. Over time, states switched to an at-large method. By 1836, no states used the district method. In 1892, the State of Michigan used the method. The reasons for this were entirely partisan. The Democratic controlled legislature wanted to boost former President Grover Cleveland's chances in the state, but felt unsure that he could win it. Cleveland recieved 5 votes, while Republican Benjamin Harrison recieved 9. In 1972, Maine changed to the district based system, and in 1992, Nebraska joined it. \\

    Apart from partisan politics, what are the benefits to a district system? A state that allocates it's electors by district has a higher change of appointing electors that accurately represent it's electorate. In the 1892 election in Michigan, had the electors been appointed to the winner, Harrison would have won all 14 electors, despite only getting 47.79\% of the vote. Had he won 99.99\%, or by only a single vote, this wouldn't have changed. This renders a significant chunk of the votes irrelevant. \\

    While the district system is a massive improvement over the at-large one, it also doesn't necessarily represent the voters. Not only is the system vulnerable to Gerrymandering, two electors must still be awarded state-wide. This contributes to an imbalance just the same as in at-large. \\

    \section{The Proposal}%%\\

    A clearer solution would be the system contained in this proposal. Under the New Electoral College, electors would be appointed state-wide, but the ratio between electors pledged to a candidate would be approximately the same as the ratio of votes cast in an election. This is achieved using a family of electoral systems, known as Proportional Representation. \\

    The first proposals for an assembly elected through proportionality was made by John Adams. Others began to speak of the idea around the turn of the eighteenth century. The first concrete proposals were introduced by Thomas Hare, an English lawyer. His original ideas included combining the entire United Kingdom into one huge electorate. He also devised the original Proportional Representation calculations, which will now be explained. Because Proportional Representation was originally designed for legislatures, that is how the explanation will be written. \\

    \say{[A legislature] should be in miniature, an exact portrait of the people at large. It should think, feel, reason, and act like them. That it may be the interest of this Assembly to do strict justice at all times, it should be an equal representation, or in other words equal interest among the people should have equal interest in it.}
    \begin{flushright}\textit{--- John Adams}\end{flushright}%%\\

    In Proportional Representation, true and exact proportionality is not possible. If all votes directly contributed towards a seat, in other words, an exact ratio, all participants would win at least one seat. Additionally, there is a finite amount of seats to be awarded, so there is a limited number of combinations of seat allocations. \\

    Hare's original method has several names, including the Hare-Niemeyer method, Largest remainder method, Hamilton's method, and Vinton's method. The Largest remainder method (LRM), is in fact, a family of methods, and the latter two are named after American Founding Father Alexander Hamilton and Representative Samuel Finley Vinton respectively, both of whom supported the method. \\

    When performing an allocation under Proportional Representation, a quota is calculated to determine how many votes will entitle a party to a seat. Hare's quota is represented as:%%\\

    \begin{align}
        \frac{\text{total valid poll}}{\text{seats}}
    \end{align}
    but more precisely:%%\\
    \begin{align}
        \left \lfloor \frac{\text{total valid poll}}{\text{seats}} \right \rfloor
    \end{align}

    The \textit{total valid poll} is the number of votes cast in an election that count. This excludes ballots where more than one vote was cast, or where no option was marked. The $\lfloor \rfloor$ markers mean the number is rounded down to the nearest integer.\\

    A party will win a seat for every time their vote count can be divided evenly into the quota. Any extra seats will be awarded to each party in descending order of the party's remainder. The following example election displays the calculation process. In this election, there are 11 seats to be filled, and a total of 1,650,799 votes are cast. Therefore, the quota is 150,072.\\

    \begin{adjustbox}{center}
    \begin{tabular}{ |l|l|l|l|l|l|l| }
        \hline
        Party & Votes & Divisor & Initial Seats & Remainder (Rank) & Extra Seats & Total Seats \\
        \hline
        A & 871,700 & 5.81 & 5 & 0.81 (2) & 1 & 6 \\
        \hline
        B & 739,894 & 4.93 & 4 & 0.93 (1) & 1 & 5 \\
        \hline
        C & 39,205 & 0.26 & 0 & 0.26 (3) & 0 & 0 \\
        \hline
    \end{tabular}
    \end{adjustbox}
    
    9 seats may be awarded based on the quota, leaving 2 seats remaining. Party B has the highest remainder, and gets the first extra seat. Party A has the next highest, and so also gets an extra seat. \\

    When the vote percentages are compared with the seat percentages, the results approximately align.\\
    
    \begin{adjustbox}{center}
    \begin{tabular}{ |l|l|l| }
        \hline
        Party & Vote Percentage & Seat Percentage \\
        \hline
        A & 52.8\% & 54.55\% \\
        \hline
        B & 44.82\% & 45.45\% \\
        \hline
        C & 2.37\% & 0\% \\
        \hline
    \end{tabular}
    \end{adjustbox}

    \section{An important caveat}%%\\

    In most countries, the votes are counted by party. That is the way the votes are cast, so it is most accurate to award them that way. However, this fails in the Electoral College. The following example is from the 2016 election in New York.\\

    \begin{adjustbox}{center}
    \begin{tabular}{ |l|l|l|l|l|l|l|l| }
        \hline
        Party & Nominee & Votes & Divisor & Initial Seats & Remainder (Rank) & Extra Seats & Total Seats \\
        \hline
        Democratic & Hillary Clinton & 4,379,804 & 16.45 & 16 & 0.45 & 1 & 17 \\
        \hline
        Republican & Donald Trump & 2,527,164 & 9.49 & 9 & 0.49 & 1 & 10 \\
        \hline
        Conservative & Donald Trump & 292,393 & 1.10 & 1 & 0.10 & 0 & 1 \\
        \hline
        Working Families & Hillary Clinton & 140,044 & 0.53 & 0 & 0.53 & 1 & 1 \\
        \hline
        Independence & Gary Johnson & 119,162 & 0.45 & 0 & 0.45 & 0 & 0 \\
        \hline
        Green & Jill Stein & 107,937 & 0.41 & 0 & 0.41 & 0 & 0 \\
        \hline
        Libertarian & Gary Johnson & 57,438 & 0.22 & 0 & 0.22 & 0 & 0 \\
        \hline
        Women's Equality & Hillary Clinton & 36,292 & 0.14 & 0 & 0.14 & 0 & 0 \\
        \hline
    \end{tabular}
    \end{adjustbox} \\

    New York has rather unique laws that allow a candidate to be nominated by more than one party. This is known as fusion voting. Allocating seats by party under this method is impractical, because it doesn't reflect the vote for President. The following figure shows the results calculated for the same election by candidate. In the example, the votes for each candidate is the sum of votes for all parties that nominated that candidate.\\

    \begin{adjustbox}{center}
    \begin{tabular}{ |l|l|l|l|l|l|l| }
        \hline
        Party & Votes & Divisor & Initial Seats & Remainder (Rank) & Extra Seats & Total Seats \\
        \hline
        Hillary Clinton & 4,556,124 & 17.11 & 17 & 0.11 (4) & 0 & 17 \\
        \hline
        Donald Trump & 2,819,534 & 10.59 & 10 & 0.59 (2) & 1 & 11 \\
        \hline
        Gary Johnson & 176,598 & 0.66 & 0 & 0.66 (1) & 1 & 1 \\
        \hline
        Jill Stein & 107,934 & 0.41 & 0 & 0.41 (3) & 0 & 0 \\
        \hline
    \end{tabular}
    \end{adjustbox}

    Johnson is allocated a seat that was assigned to Clinton in the party-wise election. This should be the prefered method of allocating electors, because it relfects the vote for President; The Electoral College is not a legislature, where parties are elected.\\
    
    After the allocation of seats to candidates, it is still neccisary to allocate seats to parties. The Largest-Remainder calculation is performed again, between the parties that nominated each candidate.\\

    \section{On the issue of third parties}%%\\

    The last example in the previous section raises an important point. Third parties will have a much higher chance of winning electors under The New Electoral College. This is most likely in states with large numbers of electoral votes. In California, for instance, the Libertarian Party likley would have won 1 of the states 55 electors in 2020 under The New Electoral College. They achived 1.07\% of the vote in California that year. However, they won 2.47\% of the votes in Alaska, but won no electors. Americans are more likley to vote for a third party in a hotly contested election. In 2020, the Libertarian elector in California would have been the sole elector from a third party that year, but in 2016, third parties would have won 16 electors.\\

    It is possible that The New Electoral College's property of electing more third party electors could have had a profound impact on the history of the United States. In 1992, George H. W. Bush was swept out of office by Bill Clinton 370-168. Independent candidate Ross Perot won an almost unprecidented 19\% of the national vote, but no electors. Under The New Electoral College, Clinton would have had 134 electors less. Perot would had been awarded 105 electors. The Constitution requires a Presidential candidate to have a majority of the votes in the Electoral College (currently 270). No Candidate would have achived this number in 1992.\\
    
    This leaves Perot with two choices. As he has no hope of being elected President under these circumstances, he could endorse Clinton or Bush. He could instruct the Electors he won to vote for the candidate he now prefers. The Electors would not be legally obligated to heed him, but his endorsement would likely sway many of them.\\
    
    In other countries, when no party has a majority of the seats in a legislative chamber, it is refered to as a hung parliament. The issue of a "hung Electoral College" would have to be addressed if The New Electoral College is implemented.\\
    
    Another instance of a hung Electoral College would have occured during the 2000 Presidential Election, yet another close contest. George W. Bush would have had a singe Elector lead over his rival, Al Gore, but six votes short of a majority. The remaining 11 seats were won by Green Party nominee, Ralph Nader. This is one instance of how The New Electoral College might have produced a different result than the existing system, because one might belive that Nader would be likley to choose Gore over Bush. While that isn't necessarily the case, Gore's enviornmental policy might have sat better with Nader than Bushs'.\\

    The other option for third party Candidates is to ignore their clear loss, and instruct their Electors to vote for them anyway. This would send the election to the House of Representatives. As a Presidential election is always accompanied by a House election, a newly elected House will decide the winner, perhaps one elected on the assumption that there might not be an Electoral College majority.\\

    An important question could be asked about the optics of placing so much power in the hands of smaller parties. It might sound like the Libertarians would have a huge amount of power under The New Electoral College, but that will only be so if voters choose it that way. Many voters had a difficult choice in 2016, and instead opted to vote for Gary Johnson, Jill Stein, or Evan McMullin. Johnson scored an unnaturally high 4 million votes because of this. In addition, no Presidential candidate won a majority of the popular vote in that election. That means Hillary Clinton was no more entitled to win outright than Donald Trump, thus justifying the presence of third parties.\\

    In the sense that third party voters often hold the ballance of power in an election, The New Electoral College is much like Instant Runnoff Voting (IRV), known in the US as Ranked Choice Voting (RCV). They will be able to determine the winner of the election if there is no clear victor at first glance.\\

    Notwithstanding any of the above, there is a legitamate concern about the potential instability caused by a more divided Electoral College. If one wanted to make the problem less likley, an Electoral Threshold could be used. An Electoral Threshold is a rule used by a large portion of countries that use Proportional Representation. Under this rule, parties may only be allocated seats if they cross a certain percentage of votes at the national level. This number is usually between 3 and 5\%. This restricts smaller parties that don't enjoy popular support from entering Parliament. Kazakhstan's 7\% electoral threshold is considered unnaturally high, and likley exists to keep opposition parites from gaining seats in Parliament. Even a threshold of 1\% would devastate America's third parties' chances under The New Electoral College.\\

    An Electoral Threshold is arguably undemocratic. As expressed above, it can be used to disadvantage parties that are considered "undesirable". Some more Authoritarian countries have been known to eliminate opposition parties from Parliament simply by rasing the Electoral Threshold. Others might argue that excluding votes on the basis of who they were cast for is soundly backwards.\\

    \section{Addressing Critisizems and Concerns about Proportional Representation}%%\\

    Proporitonal Representation is not perfect. There have been some concerns about varius aspects of it, but fortunatly, most of these do not apply to The New Electoral College.\\

    Proporitonal electoral systems have been known to destabelise some countries. Voters often excersize their ability to vote for any party without fear of wasting their vote by voting for parties that represent niche intrest groups, such as racial, ethnic, or religious groups. Many countries in Southern Asia have parties representing the Muslim minority. Some European countries have ethnic minority parties, like Italy's South Tyrolean People's Party, and the Latvian Russian Union. These are great for representing minorities, but are problematic when trying to form a government. Trying to appeal to such a broad set of intrests without a coordinated set of principles can be challenging.\\

    This concern does not apply to the Electoral College. While a Presidential candidate might try to appeal to a specific demographic, a candidate would hardly try to represent a minority group.\\

    Much also has to do with how the electoral laws are organized - there are countless ways to set up Proportional Representation. Most countries use states or provinces as constituencies, but some, including Israel, use a single nationwide constituency. This eliminates the local aspect of campaining. Israeli politics is notoriusly chaotic, mostly because of the difficulty in forming a strong, stable government when so many large factions are present. It is possible that the divisions in Israel's Knesset are due to the fact that there is no regionality to the elections.\\

    The second critisizem of Proportional Representation is also of no concern here. Under most Proportional Representation systems, seats are awarded to candidates on a ranked Party List. If a party is awarded six seats, the top six names on their list are chosen to be Members of Parliament. This kind of list system, known as Closed Lists, gives the power to choose representatives to the party, rather than the voter. It removes the direct link between a voter and their representative. There are Open List systems in existance, where voters may express preference for a candidate or candidates on a List, thereby overriding the party, but these are quite complicated, and not suitable for the Electoral College.\\

    In a more extreme example, parties in South Africa may strip their MPs of their seats in parliament for voting out of line with the party. This is because the law considers the seat to have been won by the party, not the member in question.\\

    Usually, the members of the Electoral College are immaterial. Some may even be family and friends of the candidates, or the candidates themselves. Voters usually aren't even informed at the polling stations who is on the slate of Electors for each party, and many make no attempt to find out.\\

    Proportional Representation has been critisized for allowing more extreme factions to gain representation. A claim that Proportional Representation enables Nazi parties terrified voters in the Canadian province of British Columbia into rejecting Proportional Representation several times in a referendum. While it could be said that exteme parties enjoy more representation under a Proportional system, it is debatable how much Proportional Representation is a fault for that. Some countries that use Proportional Representation simply have a higher number of extremist voters. It is difficult to study this, because a significant portion of the countries that use Proportional Representation are located in Europe, where Nazi parties are usually banned.\\

    \section{Representation of Minorities}%%\\

    One critisizem of the current Electoral College is that it does not represent racial minorities. Only a select number of western states have a majority-minority population, only one of which is a battleground state.\\

    Under The New Electoral College, each minority voter will have the same influence as any other voter in the state. The power is split proportionaly by race as well as by party.\\

    The New Electoral College will not eliminate the existance of battleground states, but will change which states get that title. This is because some states have Electors that change hands often depending on the national mood in each election. One example is Mississippi. In 2012, Mitt Romney and Barack Obama would have both won three Electors. In 2016, Clinton would have won two, and Trump four. In 2020, the numbers would have reverted to three Electors each for Trump and Biden. If I were to make an educated guess, that Elector likley has a lot to do with the Black vote in Mississippi. Turnout in Mississippi amoung Black voters was higher in 2020 and 2012 than in 2016.\\

\end{document}