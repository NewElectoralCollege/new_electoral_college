\documentclass{article}
\usepackage{bills}
\title{The New Electoral College}
\author{David Boers}

\begin{document}
    \type{joint resolution}
    \subject{To propose an Amendment to the Constitution of the United States to modify the Electoral College and for other purposes.}
    \formula{True}
    \unnamedsection{1}
    That the following article is proposed as an amendment to the Constitution of the United States, which shall be valid to all intents and purposes as part of the Constitution when ratified by the legislatures of three-fourths of the several States:
    "Article -
    1: The twenty-third article of amendment to this Constitution is hereby repealed.
    2: For the purposes of appointing Electors for President and Vice President of the United States, and for members of the House of Representatives, all territory of the United States not incorporated into any of them shall constitute an entity that shall be allocated Representatives and Electors as though it were a State. Congress, by concurrence of two-thirds of both Houses, shall determine the manner of choosing Representatives by this entity. The electors shall be in addition to those appointed by the states, but they shall be considered, for the purposes of the election of President and Vice President, to be electors appointed by a state; and they shall meet in the District constituting the seat of government of the United States and perform such duties as provided by the twelfth article of amendment.
    3: Congress shall determine, by concurrence of two-thirds of both Houses, the general manner in which the States and the territorial entity shall appoint Electors for President and Vice President. But the State Legislatures shall retain the power to oversee and manage the appointment of Electors in their respective State. 
    4: Should this article be enacted within 7 months preceding a scheduled meeting of the Electors, the States shall determine the manner of choosing Electors on that occasion.”.
    \unnamedsection{2}
    Should the Article proposed in Section 1 be enacted, each state shall appoint Electors for President and Vice President by Popular Election. 
    Either of the following entities may submit, to the electoral officer of a State, an Electoral List of candidates for Electors, and the persons nominated for President and Vice President by that List-
    An organization designated as a political party by the electoral officer of a State, or;
    A campaign committee for an independent candidate for President.
    Such an Electoral List shall rank a number of people equal to the number of Electors that the state in which the List is submitted is entitled to. 
    The electoral officer of the State shall place on the ballot such Lists as directed by law or regulation by that State. 
    Electors shall first be allocated to candidates in the following manner-
    The number of votes cast for each candidate shall be the sum of the number of votes cast for all Lists that nominated that candidate for President.
    The total number of valid votes cast for Lists in the State divided by the total number of Electors to which the state is entitled, rounded down to the nearest integer, shall constitute an electoral quota.
    A candidate shall be awarded an Elector for every time that their vote total can be divided into the quota.
    After the calculation described in subsection (3) is performed, the number of votes for each candidate shall be diminished by the product of the quota and the number of Electors awarded to that candidate in subsection (3).
    Electors that can not be awarded by the method described in subsection (3) shall be awarded to candidates in descending order of their vote total after the alteration described in subsection (4).
    Electors won by each candidate shall be distributed between the Lists that nominated that candidate, using the same calculation described in subsection (d), but where the quota is the number of votes cast for that candidate, divided by the number of Electors won by that candidate, rounded down to the nearest integer.
    If there are multiple Lists or candidates with the same number of votes after the alteration described in subsection (d)(4), but not all may receive an extra Elector, the Lists or candidates with the highest number of votes without subsection (d)(4) shall take priority. But if there are multiple Lists or candidates with equal numbers of votes, and not all can be awarded an Elector, the Lists shall be ordered by date of registration, with the Lists registered earliest taking priority. 
    Electors shall be appointed from each List, in the number of which that List is entitled, in increasing order of the rank on the List. 
    States may enact such regulations regarding the election for Electors as they deem appropriate, and that does not  –
    require an Elector to cast their vote for any specific candidate for President or Vice President; or
    contradict any provision of this Section (2).
    \unnamedsection{3}
    Should the Article proposed in Section 1 be enacted, and a State shall have joined the union within 7 months proceeding a scheduled meeting of the Electors, the Electors from that state shall be chosen in the following manner-
    The most numerous branch of the Legislature thereof shall appoint the Electors, on the first Tuesday following the first Monday in November. 
    Any other branch of the State Legislature shall approve the list of Electors.
    \unnamedsection{4}
    Should the Article proposed in Section 1 be enacted, Congress hereby prescribes the following directions for choosing Representatives for the entity described in Section 2 of the Amendment proposed in Section 1. 
    The number of votes cast for all Candidates of the same party shall be recorded.
    Seats shall be awarded to parties in the manner described in Subsection (2)(d).
    Seats shall be awarded to candidates of a party in descending order of the number of votes received by candidates of that party. 
    \unnamedsection{5}
    Section 25a of title 2, chapter 2, United States Code is hereby repealed.
    \namedsection{6}{To allow the creation of multi-member districts}
    Section 2c of title 2, chapter 1, United States Code is hereby repealed.
    An Electoral System employed by a State to elect Representatives in districts from where more than one Representative is to be elected shall involve voters casting their vote for an individual candidate, not a political party.

\end{document}