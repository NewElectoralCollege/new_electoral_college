\documentclass{article}
\usepackage{bills}
\title{The New Electoral College proposed legislation}
\author{David Boers}

\begin{document}
    \type{joint resolution}
    \subject{To propose an Amendment to the Constitution of the United States to modify the Electoral College and for other purposes.}
    \formula{True}
    \unnamedsection{1}
    That the following article is proposed as an amendment to the Constitution of the United States, which shall be valid to all intents and purposes as part of the Constitution when ratified by the legislatures of three-fourths of the several States:\\
    \begin{center}"Article -\end{center}
    1: Congress shall determine, with concurrence of two-thirds of both Houses, the general manner in which the States shall 
        appoint Electors for President and Vice President. But the State Legislatures shall retain the power to oversee and manage the appointment of 
        Electors in their respective State.\\
    2: Should this article be enacted within 7 months preceding a scheduled meeting of the Electors, the States shall determine the manner of choosing 
        Electors on that occasion.".\\
    \namedsection{2}{To prescribe the method of appointing electors under the proposed article}
    \begin{enumerate}
        \item Should the Article proposed in Section 1 be enacted, each state shall appoint Electors for President and Vice President by Popular Election.
    \end{enumerate}
    \namedsection{3}{Allocation of Electors to candidates}
    \begin{enumerate}
        \item Electors shall first be allocated to candidates in the following manner---
        \begin{enumerate}
            \item The number of votes cast for each candidate shall be the sum of the number of votes cast for all Lists that nominated that candidate for President.
            \item The total number of valid votes cast for Lists in the State divided by the total number of Electors to which the state is entitled, rounded down to the nearest integer, shall constitute an electoral quota.
            \item Candidates shall be awarded an initial number of Electors equal to the number of times that the number of votes cast for them can be equally divided by the quota.
            \item After the calculation described in subsection (a)(3) is performed, the number of votes for each candidate shall be diminished by the product of the quota and the number of Electors awarded to that candidate in subsection (a)(3).
            \item Electors that can not be awarded by the method described in subsection (a)(3) shall be awarded to candidates in descending order of their vote total after the alteration described in subsection (a)(4).
            \item Should any tie occur between two or more candidates such that it can not be determined who should be awarded an Elector, the tie shall be resolved in accordance with State law.
        \end{enumerate}
    \end{enumerate}
    \namedsection{4}{Allocation of Electors to Lists}
    \begin{enumerate}
        \item Electors won by each candidate shall be distributed between the Lists that nominated that candidate, using the same calculation described in section 3, but where the quota is the number of votes cast for that candidate, divided by the number of Electors won by that candidate, rounded down to the nearest integer.  
    \end{enumerate}
    \namedsection{5}{Electoral Lists}
    \begin{enumerate}
        \item In the Popular Election for Electors, voters shall cast their vote for a List of Elector Candidates.
        \item The number of Elector Candidates on an Electoral List shall not be less than the number of Electors to which the State in which the List is sumbitted is entitled.
        \item An Electoral List may represent---
        \begin{enumerate}
            \item An organization designated as a political party by the electoral officer of the State;
            \item A campaign committee for an independent candidate for President; or
            \item An alliance of political parties.
        \end{enumerate}
        \item Electoral Lists shall be submitted to the electoral officer of the State in which the List is running not later than one month prior to the polling date for Electors set by the respective State.
        \item The electoral officer of a State shall scrutinize the submitted Electoral Lists, and may strike Elector Candidates from them for any of the following reasons---
        \begin{enumerate}
            \item being listed on multiple Lists in the State;
            \item the wish of the Elector Candidate;
            \item the wish of the entity that submitted the List;
            \item the identity of the intended Elector Candidate being non-discernible; or
            \item non-conformance with State law regarding Elector Candidates.
        \end{enumerate}
        \item If Elector Candidates being struck from an Electoral List results in the number of Elector Candidates on the List falling below the number specified in subsection (b), the entity that submitted the Elector List shall be entitled to submit replacement Elector Candidates such that the number of Elector Candidates on the List is valid under subsection (b).
        \item Each Elector Candidate on a List shall be given a rank by the submitting entity, and Electors shall be chosen from the List in the number that the List is entitled, and in ascending order of the ranks of the Elector Candidates; no two Elector Candidates to be awarded the same rank on the List.
        \item A List shall declare the candidates it has endorsed for President and Vice President, and this shall be listed on the ballot; but a List may neglect to endorse a candidate for either office.
        \item If an Elector resigns before casting their vote for President or Vice President, dies, or is otherwise incapacitated according to State law, their place shall be filled by the next Elector Candidate on the List.
        \item No State may make any law requring an Elector to vote for any candidate for President or Vice President.
    \end{enumerate}
    \namedsection{6}{To provide an alternate method of choosing electors}
    \begin{enumerate}   
        \item Should the Article proposed in Section 1 be enacted, and a State shall have joined the union within 3 months proceeding a scheduled meeting of the Electors, the Electors from that state shall be chosen in the following manner---
        \begin{enumerate}
            \item The most numerous branch of the Legislature thereof shall appoint the Electors, on the first Tuesday following the first Monday in November.
            \item Any other branch of the State Legislature shall approve the list of Electors.
        \end{enumerate}
    \end{enumerate}
\end{document}